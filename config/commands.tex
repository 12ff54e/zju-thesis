% common commands

\ifthenelse{\equal{\TwoSide}{true}}
{
    % TwoSide settings
    % Use default `\cleardoublepage'
}
{
    % OneSide settings
    \renewcommand{\cleardoublepage}{\clearpage}
}

\newcommand{\bful}[1]{{\bfseries\uline{#1}}}

% Commands to input body pages
\ifthenelse{\equal{\Degree}{undergraduate}}
{
    % undergraduate
    \newcommand{\inputpage}[2]
    {
        % Arguments
        %    #1 [final|proposal]
        %    #2 [filename.tex]
        % Directory structure:
        % graduate/undergraduate            [Degree     ] [required        ]
        % |-- major
        %     |-- general/cs/math/...       [MajorFormat] [optional        ]
        %         |-- cover.tex/toc.tex/... [TeX Files  ] [template defined]
        %
        % Search from bottom-level to top-level:
        % E.g:
        %     With such dir:
        %     undergraduate
        %     |-- cover.tex
        %     |-- toc.tex
        %     |-- abstract.tex
        %     |-- major
        %         |-- ee
        %             |-- abstract.tex
        %     The input files are:
        %         undergraduate/cover.tex
        %         undergraduate/toc.tex
        %         undergraduate/major/ee/abstract.tex
        \IfFileExists{./page/undergraduate/#1/major/\MajorFormat/#2}
        {
            \input{./page/undergraduate/#1/major/\MajorFormat/#2}
        }
        {
            \input{./page/undergraduate/#1/#2}
        }
    }
    \newcommand{\inputbody}[1]{\input{./body/undergraduate/#1}}

    % TODO: how to switch to \section* without breaking current formats?
    \newcommand{\chapternonum}[1]
    {
        \cleardoublepage
        \phantomsection
        \addcontentsline{toc}{chapter}{#1}
        \begin{center}
            \bfseries \zihao{3} #1	
        \end{center}	
        \setcounter{section}{0}
    }

    \newcommand{\sectionnonum}[2][openright]
    {

        \ifthenelse{\equal{\TwoSide}{true}}{
            \ifthenelse{\equal{#1}{openright}}
                {\cleardoublepage}
                {\ifthenelse{\equal{#1}{openany}}{\clearpage}{}}
        }{
            \ifthenelse{\equal{#1}{openright}}
                {\clearpage}
                {\ifthenelse{\equal{#1}{openany}}{\clearpage}{}}
        }
        \phantomsection
        \addcontentsline{toc}{section}{#2}
        \begin{center}
            \bfseries \zihao{3} #2	
        \end{center}	
        \setcounter{subsection}{0}
    }
}
{
    % graduate
    \newcommand{\checkandinput}[2]
    {
        % Directory structure:
        % graduate/undergraduate            [Degree     ] [required        ]
        % |-- master/doctor                 [GradLevel  ] [optional        ]
        %     |-- general/cs/math/...       [MajorFormat] [optional        ]
        %         |-- cover.tex/toc.tex/... [TeX Files  ] [template defined]
        %
        % Search from bottom-level to top-level:
        % E.g:
        %     With such dir:
        %     graduate
        %     |-- cover.tex
        %     |-- toc.tex
        %     |-- abstract.tex
        %     |-- master
        %         |-- toc.tex
        %         |-- math
        %             |-- abstract.tex
        %     The input files are:
        %         graduate/cover.tex
        %         graduate/master/toc.tex
        %         graduate/master/math/abstract.tex
        \IfFileExists{./#1/graduate/\GradLevel/\MajorFormat/#2}
        {
            \input{./#1/graduate/\GradLevel/\MajorFormat/#2}
        }
        {
            \IfFileExists{./#1/graduate/\GradLevel/#2}
            {
                \input{./#1/graduate/\GradLevel/#2}
            }
            {
                \input{./#1/graduate/#2}
            }
        }
    }

    \newcommand{\inputpage}[1]
    {
        \checkandinput{page}{#1}
    }
    
    
    \newcommand{\inputbody}[1]
    {
        \checkandinput{body}{#1}
    }

    \newcommand{\chapternonum}[1]
    {
        \phantomsection
        \addcontentsline{toc}{chapter}{#1}
        \markboth{#1}{#1}
        \chapter*{#1}
    }

    \newcommand{\sectionnonum}[1]
    {
        \phantomsection
        \addcontentsline{toc}{section}{#1}
        \section*{#1}
    }
}

\ifthenelse{\equal{\Degree}{undergraduate}}
{
\newcommand{\signature}[1]
{
    \begin{flushright}
        \bfseries \zihao{-4}
        #1 \underline{\multido{}{5}{\quad}} \\
        \quad 年 \quad 月 \quad 日
    \end{flushright}
}

\DeclareDocumentCommand{\finaleval}{O{~} O{~} O{~} O{~} O{~}}
{
    \begin{table}[H]
        \centering \bfseries
        \begin{tabularx}{\textwidth}{|>{\fangsong}c
                                     |>{\fangsong}X<{\centering}
                                     |>{\fangsong}X<{\centering}
                                     |>{\fangsong}X<{\centering}
                                     |>{\fangsong}X<{\centering}
                                     |>{\fangsong}c|}
            \hline
            \makecell{成绩\\比例}
            & \makecell{\ifthenelse{\equal{\Type}{thesis}}{文献综述}{中期报告} \\(10\%)}
            & \makecell{开题报告\\(15\%)}
            & \makecell{外文翻译\\(5\%)}
            & \ifthenelse{\equal{\Type}{thesis}}{毕业论文质量及答辩(70\%)}{毕业设计质量及答辩(70\%)}
            & \makecell{总评\\成绩} \\

            \hline
            \multirow{2}*{分值}
            & \multirow{2}*{\zihao{4}#1}
            & \multirow{2}*{\zihao{4}#2}
            & \multirow{2}*{\zihao{4}#3}
            & \multirow{2}*{\zihao{4}#4}
            & \multirow{2}*{\zihao{4}#5} \\

            ~ & ~ & ~ & ~ & ~ & ~ \\
            \hline
        \end{tabularx}
    \end{table}
}

% `design` commands
\DeclareDocumentCommand{\designproposaleval}{O{~} O{~}}
{
    \begin{flushright}
        \begin{tabular}{| >{\fangsong \zihao{4}}c
                        | >{\fangsong \zihao{5}}c
                        | >{\fangsong \zihao{5}}c |}
            \hline
            \multirow{2}*{成绩比例}
            & 开题报告
            & 外文翻译 \\

            ~
            & 占(15\%)
            & 占(5\%) \\

            \hline

            \multirow{2}*{分值}
            & \multirow{2}*{\zihao{4}#1}
            & \multirow{2}*{\zihao{4}#2} \\
            
            ~
            & ~
            & ~ \\
            \hline
        \end{tabular}
    \end{flushright}
}

\DeclareDocumentCommand{\designmidcheckeval}{O{~}}
{
    \begin{flushright}
        \begin{tabular}{| >{\fangsong \zihao{4}}c
                        | >{\fangsong \zihao{5}}c |}
            \hline
            \multirow{2}*{成绩比例}
            & 中期报告 \\

            ~
            & (10\%) \\

            \hline

            \multirow{2}*{分值}
            & \multirow{2}*{\zihao{4}#1} \\

            ~
            & ~ \\
            \hline
        \end{tabular}
    \end{flushright}
}

% `thesis` commands

% `thesis` commands
\DeclareDocumentCommand{\thesisproposaleval}{O{~} O{~} O{~}}
{
    \begin{flushright}
        \begin{tabular}{| >{\fangsong \zihao{4}}c
                        | >{\fangsong \zihao{5}}c
                        | >{\fangsong \zihao{5}}c
                        | >{\fangsong \zihao{5}}c |}
            \hline
            \multirow{2}*{成绩比例}
            & 文献综述
            & 开题报告
            & 外文翻译 \\

            ~
            & 占(10\%)
            & 占(15\%)
            & 占(5\%) \\

            \hline

            \multirow{2}*{分值}
            & \multirow{2}*{\zihao{4}#1}
            & \multirow{2}*{\zihao{4}#2}
            & \multirow{2}*{\zihao{4}#3} \\

            ~
            & ~
            & ~
            & ~ \\
            \hline
        \end{tabular}
    \end{flushright}
}
}
{}

% From: https://tex.stackexchange.com/questions/395856/switching-tocdepth-in-the-middle-of-a-document
\newcommand{\changelocaltocdepth}[1]{%
  \addtocontents{toc}{\protect\setcounter{tocdepth}{#1}}%
  \setcounter{tocdepth}{#1}%
}